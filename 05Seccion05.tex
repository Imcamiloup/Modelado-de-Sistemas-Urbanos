

\chapter{Datos y preprocesamiento}



\section*{Análisis de Variables para el Modelo Matemático}

A continuación, se presenta un análisis de las variables necesarias para el modelo matemático, evaluando la factibilidad de conseguir los datos.

\begin{table}[ht]
\centering
\begin{tabular}{|p{3cm}|p{3cm}|p{4cm}|p{4cm}|}
\hline
\textbf{Variable} & \textbf{Frecuencia de Actualización} & \textbf{Descripción} & \textbf{Fuente de Datos} \\
\hline
\textbf{Población (P)} & Anual/Intercensal & Número de habitantes en Bogotá. & DANE, censos nacionales, proyecciones gubernamentales \\
\hline
\textbf{Huella Urbana (H)} & Anual o periódica & Extensión territorial ocupada por la ciudad (área construida y ocupada). & Planes de ordenamiento territorial, imágenes satelitales, mapas urbanos. \\
\hline
\textbf{Calidad del Aire (A)} & Mensual & Niveles de contaminación atmosférica (PM2.5, PM10, NOx, CO2). & Secretaría de Ambiente, estaciones de monitoreo de calidad del aire. \\
\hline
\textbf{Estructura Ecológica (E)} & Anual & Zonas verdes, áreas protegidas, corredores ecológicos. & Instituto Distrital de Medio Ambiente (IDRD), estudios ecológicos. \\
\hline
\textbf{Infraestructura (I)} & Anual & Servicios básicos (agua, electricidad, alcantarillado, transporte) y su distribución en la ciudad. & Empresas de servicios públicos, censos de infraestructura urbana. \\
\hline
\textbf{Bienestar Social (B)} & Anual & Índices de acceso a servicios básicos (salud, educación, vivienda) y calidad de vida. & Encuestas sociales, estadísticas de salud, educación y vivienda. \\
\hline
\textbf{Consumo de Recursos (C)} & Anual & Consumo de agua, energía, alimentos, entre otros. & Empresas de servicios, estudios de consumo energético. \\
\hline
\end{tabular}
\caption{Análisis de Variables: Frecuencia, Descripción y Fuente de Datos}
\end{table}

\end{document}

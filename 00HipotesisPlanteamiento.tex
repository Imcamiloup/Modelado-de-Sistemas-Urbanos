\chapter{Estado del Arte}

El uso de \textbf{sistemas de ecuaciones diferenciales ordinarias (EDOs)} para modelar ciudades sostenibles ha demostrado ser eficaz al capturar interdependencias entre población, recursos, infraestructura y calidad ambiental. La resolución de estas EDOs puede requerir del uso de \textit{técnicas de análisis numérico}, y de optimización, permitiendo simular con precisión escenarios de evolución de sistemas urbanos complejos. Asimismo, la calibración de los parámetros del modelo se fundamenta en \textit{algoritmos de optimización lineal y no lineal}, que ajustan las ecuaciones a los datos empíricos locales para garantizar que las trayectorias simuladas reproduzcan fielmente la dinámica observada \cite{Nocedal2006,Boyd2004}.

Un antecedente fundacional es el modelo \textbf{Urban Dynamics}, desarrollado por \citeauthor{forrester1970} (\citeyear{forrester1970}), que representó la ciudad como un conjunto de \textit{stocks y flujos interconectados}: población, empleo, vivienda e inversión pública. Este enfoque inauguró la tradición de modelar la ciudad como un \textit{sistema dinámico con retroalimentaciones positivas y negativas}, capaz de explicar fases de crecimiento, madurez y declive urbano. La metodología de Urban Dynamics constituye una base sólida y justifica el uso de \textit{variables agregadas}, la formulación de \textit{ecuaciones diferenciales acopladas}, y el análisis de escenarios para apoyar decisiones de política urbana.

En la misma línea, un caso emblemático es el modelo \textbf{Wonderland}, ideado por \citeauthor{sanderson1994simulation} 
(\citeyear{sanderson1994simulation}), que emplea cuatro variables continuas (población, producción per cápita, capital natural y contaminación) para explorar futuros extremos: el “Sueño” (crecimiento sostenible indefinido) y el “Horror” (colapso ecológico y extinción). Este apelativo enfatiza la dualidad de escenarios deseables y catastróficos. Su interés radica en identificar regiones paramétricas que generan sostenibilidad o colapso y en exhibir comportamientos caóticos, lo cual refuerza los objetivos de realizar \textit{análisis de sensibilidad, evaluación de estabilidad} y el uso de \textit{software matemático para simulación} \cite{Strogatz2014,Saltelli2008}.

En vista de la crisis climática e inminente escasez de agua en todo el mundo causado por el cambio climatico, muchas ciudades e instituciones han comenzado a crear practicas ciudadanas junto con modelos matemáticos del \textbf{ciclo urbano del agua}, estos modelos han adoptado técnicas basadas en EDOs para formular balances continuos de \textit{infiltración, almacenamiento y drenaje pluvial}. En el marco de \textit{Water Sensitive Urban Design (WSUD)} o “ciudades esponja”, se han aplicado ecuaciones diferenciales para representar procesos hídricos urbanos y evaluar cómo la infraestructura verde-azul contribuye a la resiliencia frente a inundaciones y al uso eficiente del recurso hídrico \cite{Wong2006}, también se han realizado modelos de EDOs en el marco latinoamericano diseñados por \citeauthor{gutierrez2023modelo} 
(\citeyear{gutierrez2023modelo}) , donde se justifica la importancia de mantener suministros de agua en cantidad y calidad suficientes para garantizar sostenibilidad del agua a pesar de los diversos efectos causados por el cambio climatico.



En conjunto, estos antecedentes muestran que la combinación de enfoques clásicos como \textbf{Urban Dynamics} y \textbf{Wonderland}, junto con aplicaciones sectoriales como WSUD y el uso de \textit{técnicas de matemáticas aplicadas, análisis numérico y optimización}, conforman una base metodológica robusta para desarrollar un \textbf{modelo matemático de sostenibilidad urbana en Bogotá} \citeauthor{elsawah2017overview} 
(\citeyear{elsawah2017overview}). Este proyecto se apoyará en estos aportes junto con un sólido uso de EDOs, lo que permitirá articular \textit{variables clave} (población, agua, residuos y servicios, entre otros), calibrarlas con datos locales y explorar escenarios de \textit{sostenibilidad y colapso} en el horizonte de los próximos diez años.

\chapter{Metodologia y métodos de análisis}


\section{Función compuesta de desigualdad F(t)}

\subsection{Definición del indicador}
Sea $t$ el año calendario. Definimos la función compuesta de desigualdad como una combinación lineal de tres componentes observables:
\begin{equation}
  F(t) \;=\; \alpha_{1}\,\frac{IPM_t}{100} \;+\; \alpha_{2}\,Gini_t \;+\; \alpha_{3}\,\frac{1}{IPC_t},
  \label{eq:Ft}
\end{equation}
donde:
\begin{itemize}
  \item $IPM_t$: Índice de Pobreza Monetaria en $t$ (en puntos porcentuales).
  \item $Gini_t$: coeficiente de Gini en $t$.
  \item $IPC_t$: indicador monetario per cápita en $t$ (magnitud positiva).
\end{itemize}
La división $IPM_t/100$ normaliza la escala de $IPM$ al intervalo $[0,1]$; el término $1/IPC_t$ captura el efecto inverso de mayores niveles per cápita.

\subsection{Datos}
Se empleó la tabla \texttt{Indice\_total\_de\_pobreza.xlsx}, con variables $\{IPM_t, Gini_t, IPC_t\}$ para los años disponibles. Todos los campos se convirtieron a numéricos y se ordenaron temporalmente.

\subsection{Criterio de estimación}
Dados valores observados de una referencia empírica $F_{\text{real},t}$ (serie objetivo), los parámetros $\boldsymbol{\alpha}=(\alpha_1,\alpha_2,\alpha_3)$ se estiman por mínimos cuadrados no restringidos:
\begin{equation}
  \hat{\boldsymbol{\alpha}}
  \;=\;
  \arg\min_{\alpha_1,\alpha_2,\alpha_3}
  \sum_{t} \Big( F_{\text{real},t} \;-\; \alpha_{1}\,\tfrac{IPM_t}{100} \;-\; \alpha_{2}\,Gini_t \;-\; \alpha_{3}\,\tfrac{1}{IPC_t} \Big)^{2}.
  \label{eq:argmin}
\end{equation}
El problema (\ref{eq:argmin}) se resolvió con el método de Nelder–Mead a partir de una semilla $(0.5,\,0.3,\,0.2)$, sobre los datos limpios de la tabla.

\subsection{Parámetros estimados}
La optimización produjo los siguientes coeficientes:
\begin{align*}
  \hat{\alpha}_{1} &= 8.3725\times10^{-8},\\
  \hat{\alpha}_{2} &= 1.0000008,\\
  \hat{\alpha}_{3} &= -0.4916743.
\end{align*}

\subsection{Construcción final de la serie}
Sustituyendo los coeficientes en (\ref{eq:Ft}), la serie compuesta se obtiene, para cada año $t$, como
\begin{equation}
  \widehat{F}(t) \;=\; \hat{\alpha}_{1}\,\frac{IPM_t}{100} \;+\; \hat{\alpha}_{2}\,Gini_t \;+\; \hat{\alpha}_{3}\,\frac{1}{IPC_t}.
  \label{eq:Ft-final}
\end{equation}
Esta expresión define el cálculo operativo de $F(t)$ a partir de las variables disponibles.
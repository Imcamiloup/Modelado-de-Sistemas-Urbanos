\newpage
\thispagestyle{empty}
\chapter*{\sffamily Resumen}
\addcontentsline{toc}{chapter}{Resumen}%
\begin{center}
\textbf{\large \thesisname}
\end{center}
\par 
Este trabajo de grado propone un modelo matemático basado en ecuaciones diferenciales ordinarias (EDOs) para analizar la sostenibilidad urbana de Bogotá, integrando variables clave como la población, la infraestructura urbana, los servicios públicos y los factores ecológicos que afectan la calidad ambiental y el bienestar social. Siguiendo las bases teóricas del modelo **Urban Dynamics** de Forrester (1970) y el modelo **Wonderland** de Sanderson (1994), se utiliza un enfoque de **modelado dinámico** para capturar la interacción de estos componentes y simular escenarios de evolución urbana y posibles transiciones hacia regímenes de sostenibilidad o colapso.

El modelo comienza con la representación de población, empleo, vivienda e inversión pública, tomando en cuenta las retroalimentaciones positivas y negativas en el sistema urbano. A través de ecuaciones acopladas, se simulan las dinámicas de crecimiento y declive urbano, permitiendo evaluar políticas de gestión urbana en función de los umbrales de estabilidad del sistema.

Una vez abordada la dinámica básica urbana, el modelo se enfoca en el análisis de la gestión de residuos sólidos. Se representa la acumulación de basura en los rellenos sanitarios, modelando los flujos de residuos a lo largo de su ciclo de vida, desde la generación hasta la disposición final. El modelo analiza los impactos ecológicos de los rellenos, evaluando cómo las tasas de reciclaje y disposición final afectan la calidad ambiental y la salud urbana, considerando la capacidad de los rellenos y su relación con el entorno circundante. Este análisis es clave para prever posibles colapsos ecológicos y determinar políticas eficientes para la gestión de residuos.

Finalmente, el modelo incorpora la gestión del agua como fase final del desarrollo del modelo de sostenibilidad. Basado en el enfoque de **Water Sensitive Urban Design (WSUD)**, se utilizan EDOs para simular procesos clave del ciclo urbano del agua, tales como infiltración, almacenamiento y drenaje pluvial. Este submodelo evalúa el impacto de las políticas de infraestructura verde y azul, como espacios verdes urbanos, jardines de lluvia y sistemas de recogida de agua de lluvia, en la resiliencia hídrica de la ciudad frente a inundaciones y sequías. Se exploran diferentes escenarios climáticos y de crecimiento urbano para identificar soluciones sostenibles que garanticen el uso eficiente del recurso hídrico.

El modelo se implementa en Python, utilizando herramientas de optimización numérica para calibrar los parámetros con datos locales y generar gráficos y diagramas de fase que permiten visualizar las interacciones entre las variables del sistema. A través de simulaciones de escenarios de políticas urbanas, como la gestión de residuos, la infraestructura verde y el uso eficiente del agua, se evalúan las trayectorias futuras del sistema urbano de Bogotá y su capacidad de adaptación frente a los desafíos ambientales.




\\[2cm]
\textbf{Palabras clave:} \palabrasclave

\newpage
\thispagestyle{empty}
\chapter*{\sffamily Abstract}
\addcontentsline{toc}{chapter}{Abstract}%
\begin{center}
\textbf{\large \thesisnameeng}
\end{center}
\par This thesis proposes a mathematical model based on ordinary differential equations (ODEs) to analyze the urban sustainability of Bogotá, integrating key variables such as population, urban infrastructure, public services, and ecological factors affecting environmental quality and social well-being. Based on the theoretical foundations of Forrester's **Urban Dynamics model (1970)** and Sanderson's **Wonderland model (1994)**, a dynamic modeling approach is used to capture the interaction of these components and simulate urban evolution scenarios, including potential transitions to sustainability or collapse regimes.

The model begins by representing population, employment, housing, and public investment, accounting for positive and negative feedbacks in the urban system. Through coupled equations, it simulates growth and decline dynamics, allowing the evaluation of urban management policies based on the system's stability thresholds.

Once the basic urban dynamics are addressed, the model focuses on solid waste management. It represents the accumulation of waste in landfills, modeling waste flows from generation to final disposal. The ecological impacts of landfills are assessed, examining how recycling rates and disposal practices affect environmental quality and urban health, while considering landfill capacity and its relationship with surrounding environments. This analysis is crucial for predicting potential ecological collapse and determining efficient waste management policies.

Finally, the model integrates water management as the final phase in the urban sustainability model. Based on **Water Sensitive Urban Design (WSUD)** principles, ODEs are used to simulate key processes of the urban water cycle, such as infiltration, storage, and stormwater drainage. This submodel assesses the impact of green and blue infrastructure policies, such as urban green spaces, rain gardens, and rainwater harvesting systems, on the city's water resilience to floods and droughts. Different climatic and urban growth scenarios are explored to identify sustainable solutions that ensure efficient water use.

The model is implemented in Python, using numerical optimization tools to calibrate parameters with local data and generate graphs and phase diagrams that visualize the interactions between system variables. Through simulations of urban policy scenarios, such as waste management, green infrastructure, and water use efficiency, the model evaluates the future trajectories of Bogotá's urban system and its adaptation capacity to environmental challenges.

\\[2cm]
\textbf{Keywords:} \keywords

